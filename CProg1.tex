\documentclass[10pt,t]{beamer}


\setbeamersize{text margin left=10pt,text margin right=10pt}
\usetheme{lehigh}

\usefonttheme{professionalfonts}
\usefonttheme{serif}

%\usepackage{pgf,pgfarrows,pgfnodes,pgfautomata,pgfheaps,pgfshade}
%\usepackage{amsmath,amssymb,amsfonts,subfigure,pifont}
\usepackage{multirow,multicol}
%\usepackage{tabularx}
%\usepackage{booktabs}
\usepackage{colortbl}
\usepackage{algorithm,algpseudocode}
%\usepackage{etex}
\usepackage{fancyvrb,listings}

\definecolor{dkgreen}{rgb}{0,0.6,0}
\definecolor{grey}{rgb}{0.5,0.5,0.5}
\definecolor{mauve}{rgb}{0.58,0,0.82} 
\lstset{%
language=bash,                % the language of the code
%basicstyle=\footnotesize,           % the size of the fonts that are used for the code
basicstyle=\fontsize{4.5}{5.5}\selectfont\ttfamily,
showspaces=false,               % show spaces adding particular underscores
showstringspaces=false,         % underline spaces within strings
showtabs=false,                 % show tabs within strings adding particular underscores
%frame=single,                   % adds a frame around the code
%rulecolor=\color{black},        % if not set, the frame-color may be changed on line-breaks within not-black text (e.g. comments (green here))
tabsize=2,                      % sets default tabsize to 2 spaces
%captionpos=b,                   % sets the caption-position to bottom
breaklines=true,                % sets automatic line breaking
breakatwhitespace=false,        % sets if automatic breaks should only happen at whitespace
%title=\lstname,                   % show the filename of files included with \lstinputlisting;
% also try caption instead of title
keywordstyle=\color{blue},          % keyword style
commentstyle=\color{dkgreen},       % comment style
stringstyle=\color{mauve},         % string literal style
escapeinside={\%*}{*)},            % if you want to add LaTeX within your code
morekeywords={*,\dots,elif},              % if you want to add more keywords to the set
deletekeywords={\dots},              % if you want to delete keywords from the given language
%morecomment=[l]{//}
}
\lstset{%
language=csh,                % the language of the code
%basicstyle=\footnotesize,           % the size of the fonts that are used for the code
basicstyle=\fontsize{4.5}{5.5}\selectfont\ttfamily,
showspaces=false,               % show spaces adding particular underscores
showstringspaces=false,         % underline spaces within strings
showtabs=false,                 % show tabs within strings adding particular underscores
%frame=single,                   % adds a frame around the code
%rulecolor=\color{black},        % if not set, the frame-color may be changed on line-breaks within not-black text (e.g. comments (green here))
tabsize=2,                      % sets default tabsize to 2 spaces
captionpos=b,                   % sets the caption-position to bottom
breaklines=true,                % sets automatic line breaking
breakatwhitespace=false,        % sets if automatic breaks should only happen at whitespace
%title=\lstname,                   % show the filename of files included with \lstinputlisting;
% also try caption instead of title
keywordstyle=\color{blue},          % keyword style
commentstyle=\color{dkgreen},       % comment style
stringstyle=\color{mauve},         % string literal style
escapeinside={\%*}{*)},            % if you want to add LaTeX within your code
morekeywords={*,\dots,elif},              % if you want to add more keywords to the set
deletekeywords={\dots},              % if you want to delete keywords from the given language
%morecomment=[l]{//}
}
\lstdefinestyle{LINUX}
{
%    backgroundcolor=\color{black},
    basicstyle=\tiny\ttfamily,
%    keywordsstyle=\color{blue},
%    morekeywords={Tutorials,BASH,scripts},
%    literate={>}{{\textcolor{blue}{>}}}1
%         {/}{{\textcolor{blue}{/}}}1
%         {./}{{\textcolor{black}{./ }}}1
%         {~}{{\textcolor{blue}{\textasciitilde}}}1,
}

\lstdefinestyle{customc}{
  belowcaptionskip=1\baselineskip,
  breaklines=true,
  xleftmargin=\parindent,
  language=C,
  showstringspaces=false,
  basicstyle=\footnotesize\ttfamily,
  keywordstyle=\bfseries\color{green!40!black},
  commentstyle=\upshape\color{red!90!white},
  identifierstyle=\color{blue},
  stringstyle=\color{orange},
}
\lstdefinelanguage{OmpFortran}[]{Fortran}{
   rulesepcolor=\color{black},
   %
   extendedchars=true,
   %
   morecomment=[l] [\bfseries\color{red!90!white}]{!\$omp},
   morecomment=[l] [\bfseries\color{red!90!white}]{c\$omp},
   morecomment=[l] [\bfseries\color{red!90!white}]{*\$omp},
   morecomment=[l] [\bfseries\color{red!90!white}]{!\$acc},
   morecomment=[l] [\bfseries\color{red!90!white}]{c\$acc},
   morecomment=[l] [\bfseries\color{red!90!white}]{*\$acc},
}[comments]

\lstdefinelanguage{OmpC}[]{OmpFortran}{
   rulesepcolor=\color{black},
   %
   extendedchars=true,
   %
   morecomment=[l] [\bfseries\color{red!90!white}]{\#pragma\ omp},
   morecomment=[l] [\bfseries\color{red!90!white}]{\#pragma\ acc},
}[comments]

\lstset{escapechar=@,style=customc}
\lstset{literate=%
   *{0}{{{\color{blue}0}}}1
    {1}{{{\color{blue}1}}}1
    {2}{{{\color{blue}2}}}1
    {3}{{{\color{blue}3}}}1
    {4}{{{\color{blue}4}}}1
    {5}{{{\color{blue}5}}}1
    {6}{{{\color{blue}6}}}1
    {7}{{{\color{blue}7}}}1
    {8}{{{\color{blue}8}}}1
    {9}{{{\color{blue}9}}}1
}

\algrenewcommand\algorithmicfunction{\textbf{program}}
\algblockdefx[Program]{Program}{EndProgram}[1]{\textbf{program} \textsc{#1}}[1]{\textbf{end program} \textsc{#1}}
\algloopdefx[doloop]{Do}[1]{\textbf{do} #1}
\algcblockdefx[doloop]{If}{Do}{EndDo}
[1]{\textbf{do} #1}{\textbf{end do}}


\usepackage{tikz}
\usetikzlibrary{shapes,arrows,matrix}
\usetikzlibrary{calc}
\pgfdeclarelayer{background}
\pgfdeclarelayer{foreground}
\pgfsetlayers{background,main,foreground}
\usepackage[latin1]{inputenc}
\usepackage[english]{babel}
\usepackage{hyperref}
\usepackage[normalem]{ulem}
% \usepackage{movie15} 

                                                         
\usepackage{times}
\usepackage[T1]{fontenc}
\usepackage{graphicx}


\definecolor{DarkGreen}{rgb}{0.0,0.3,0.0}
\definecolor{Blue}{rgb}{0.0,0.0,0.8} 
\definecolor{dodgerblue}{rgb}{0.1,0.1,1.0}
\definecolor{indigo}{rgb}{0.41,0.1,0.0}
\definecolor{seagreen}{rgb}{0.1,1.0,0.1}
\DeclareSymbolFont{extraup}{U}{zavm}{m}{n}
%\DeclareMathSymbol{\vardiamond}{\mathalpha}{extraup}{87}
\newcommand{\cmark}{\ding{51}}
\newcommand{\xmark}{\ding{55}}
\newcommand{\smark}{\ding{77}}
\newcommand*\vardiamond{\textcolor{lubrown}{%
  \ensuremath{\blacklozenge}}}
\newcommand*\up{\textcolor{green}{%
  \ensuremath{\blacktriangle}}}
\newcommand*\down{\textcolor{red}{%
  \ensuremath{\blacktriangledown}}}
\newcommand*\const{\textcolor{darkgray}%
  {\textbf{--}}}
\newcommand*\enter{\tikz[baseline=-0.5ex] \draw[<-] (0,0) -| (0.5,0.1);}
\newcommand{\bftt}[1]{\textbf{\texttt{#1}}}
\newcommand{\lstfortran}[1]{\lstinline[language={[90]Fortran},basicstyle=\footnotesize\ttfamily]|#1|}
\newcommand{\Verblue}[1]{\Verb[formatcom=\color{blue},commandchars=\\\{\}]!#1!}
\newcommand{\Verbindigo}[1]{\Verb[formatcom=\color{indigo},commandchars=\\\{\}]!#1!}

\setbeamercolor{uppercol}{fg=white,bg=red!30!black}%
\setbeamercolor{lowercol}{fg=black,bg=red!15!white}%
\setbeamercolor{uppercol1}{fg=white,bg=blue!30!black}%
\setbeamercolor{lowercol1}{fg=black,bg=blue!15!white}%%
\setbeamercolor{uppercol2}{fg=white,bg=green!30!black}%
\setbeamercolor{lowercol2}{fg=black,bg=green!15!white}%
\newenvironment{colorblock}[4]
{
\setbeamercolor{upperblock}{fg=#1,bg=#2}
\setbeamercolor{lowerblock}{fg=#3,bg=#4}
\begin{beamerboxesrounded}[upper=upperblock,lower=lowerblock,shadow=true]}
{\end{beamerboxesrounded}}
\newenvironment{ablock}[0]
{
\begin{beamerboxesrounded}[upper=uppercol,lower=lowercol,shadow=true]}
{\end{beamerboxesrounded}}
\newenvironment{bblock}[0]
{
\begin{beamerboxesrounded}[upper=uppercol1,lower=lowercol1,shadow=true]}
{\end{beamerboxesrounded}}
\newenvironment{eblock}[0]
{
\begin{beamerboxesrounded}[upper=uppercol2,lower=lowercol2,shadow=true]}
{\end{beamerboxesrounded}}


\beamertemplateballitem

\newcolumntype{a}{>{\columncolor{lulime}}c}
\newcolumntype{b}{>{\columncolor{lulime!60}}c}


\hypersetup{
  pdftitle={C Programming}
  pdfauthor={Alexander B. Pacheco, LTS Research Computing, Lehigh University}
}

\title{C Programming I}


\author[Alex Pacheco]{\large{Alexander~B.~Pacheco}}
       
\institute{\href{http://researchcomputing.lehigh.edu}{LTS Research Computing}}

\date{\today}
     
\subject{Talks}
\keywords{Lehigh Research Computing Resources, C Programming}
% This is only inserted into the PDF information catalog. Can be left
% out. 


% Delete this, if you do not want the table of contents to pop up at
% the beginning of each subsection:
\AtBeginSection[]
{
  \begingroup
  \setbeamertemplate{background canvas}[vertical shading][bottom=lubrown,top=lubrown]
  \setbeamertemplate{footline}[myfootline] 
  \setbeamertemplate{section page}[mysection]
  \frame[c]{
    \sectionpage
  }
  \endgroup
}

\titlegraphic{\includegraphics[scale=0.5]{lu}}

\begin{document}

\frame{\titlepage}

\begin{frame}{Outline}
  \tableofcontents
\end{frame}

\section{Introduction}
\begin{frame}{What is the C Language?}
  \begin{itemize}
    \item A general-purpose, procedural, imperative computer programming language.
    \item Developed in 1972 by Dennis M. Ritchie at the Bell Telephone Laboratories to develop the UNIX operating system.
    \item The UNIX operating system, the C compiler, and essentially all UNIX applications programs have been written in C.
    \item C is the most widely used computer language.
    \begin{itemize}
      \item Easy to learn
      \item Structured language
      \item Produces efficient programs
      \item Handles low-level activities
      \item Can be compiled on a variety of computer plaforms
    \end{itemize}
    \item Most of the state-of-the-art softwares have been implemented using C.
    \item Today's most popular Linux OS and RBDMS MySQL have been written in C.
  \end{itemize}
\end{frame}

\begin{frame}{What do you need to learn C?}
  \begin{enumerate}
    \item {C Compiler}
      \begin{itemize}
        \item What is a Compiler?
          \begin{itemize}
            \item A compiler is a computer program (or set of programs) that transforms source code written in a programming language (the source language) into another computer language (the target language, often having a binary form known as object code).
          \end{itemize}
        \item How does a compiler do?
          \begin{itemize}
            \item Translate C source code into a binary executable
          \end{itemize}
        \item List of Common Compilers:
          \begin{itemize}
            \item GCC GNU Project (Free, available on most *NIX systems)
            \item Intel Compiler
            \item Portland Group (PGI) Compiler
            \item Microsoft Visual Studio
            \item IBM XL Compiler
          \end{itemize}
      \end{itemize}
    \item {Text Editor}
      \begin{itemize}
        \item Emacs
        \item VI/VIM
        \item Notepad++ (avoid Notepad if you will eventually use a *NIX system)
        \item Integrated Development Environment: Eclipse, XCode, Visual Studio, etc
      \end{itemize}
  \end{enumerate}
\end{frame}

\section{Program Structure}
\begin{frame}[fragile]{Program Structure}
A C Program consists of the following parts
  \begin{itemize}
    \item Preprocessor Commands
    \item Functions
    \item Variables
    \item Statements \& Expressions
    \item Comments
  \end{itemize}
  \begin{columns}
    \column{0.5\textwidth}
    A Simple Hello World Code
    \lstinputlisting[language=C]{./Example/hello.c}
    \column{0.5\textwidth}
    Compile and execute the code
    \begin{lstlisting}[style=LINUX]
dyn100077:Exercise apacheco$ gcc hello.c 
dyn100077:Exercise apacheco$ ./a.out 
Hello World!
    \end{lstlisting}
  \end{columns}
\end{frame}

\begin{frame}[fragile]{My First C Code}
  \lstinputlisting[language=C,numbers=left]{./Example/hello.c}
  \begin{itemize}
    \item \lstinline[language=C]|#include <stdio.h>| is a preprocessor command.
    \item[] It tells a C compiler to include stdio.h file before going to actual compilation.
    \item \lstinline[language=C]|int main()| is the main function where program execution begins.
    \item \lstinline[language=C]|/* ... */| is a comment and ignored by the compiler.
    \item \lstinline[language=C]|printf(...)| is function that prints \lstinline[language=C]|Hello World!| to the screen.
    \item \lstinline[language=C]|return 0;| terminates main() function and returns the value 0.
  \end{itemize}
\end{frame}

\section{Basic Syntax}
\begin{frame}[fragile,allowframebreaks]{Basic C Syntax}
  \begin{itemize}
    \item C is a case sensitive programming language i.e. program is not the same as Program or PROGRAM.
    \item Each individual statement must end with a semicolon. 
    \item Whitespace i.e. tabs or spaces is insignificant except whitespace within a character string.
    \item All C statments are free format i.e. no specified layout or column assignment as in FORTRAN77.
      \begin{lstlisting}
#include <stdio.h>
int main () {  /* My First C Code */  printf("Hello World!\n");  return 0;}
      \end{lstlisting}
    \item[] will produce the exact same result as the code on the previous slide.
    \item In C everything within \lstinline|/* and */| is a comment. Comments can span multiple lines.
      \begin{lstlisting}
/* this is single line comment */
/* This
is a 
multiline comment */
      \end{lstlisting}
%      \item Always use proper comments in your code. Your code will most likely be handed to someone long after you are gone.
%      \item Comments are completely ignored by compiler (test/debug code)
  \end{itemize}
\end{frame}

\begin{frame}[fragile]{Valid Character Set in C language}
  \begin{itemize}
    \item White space Characters: blank space, new line, horizontal tab, carriage return and form feed
  \end{itemize}
  \begin{center}
    \begin{tabular}{|a|b|}
      \hline
      {Alphabets} & ABCDEFGHIJKLMNOPQRSTUVWXYZ \\
      & abcdefghijklmnopqrstuvwxyz \\
      Digits                     & 0123456789 \\
      \hline
    \end{tabular}
    \newline
    \begin{tabular}{abababababababa}
      \hline
      \multicolumn{15}{a}{Special Characters} \\
      \hline
      , & \_ & \{ & < & ' & ( & \Verb|^| & ; & \$ & / & *                & + & [ & \Verb|#| & ? \\ 
      . & \& & \} & > & " & ) & \Verb|!| & : & \% & | & \textbackslash{} & - & ] & \Verb|~| & \\
      \hline
    \end{tabular}
    \newline
    \begin{tabular}{|c|c|c|c|}
      \hline
      \multicolumn{4}{a}{Reserved Keywords}\\
      \hline
      \rowcolor{lulime!60} auto     & double & int      & struct \\
      \rowcolor{lulime}    break    & else   & long     & switch \\
      \rowcolor{lulime!60} case     & enum   & register & typedef \\
      \rowcolor{lulime}    char     & extern & return   & union \\
      \rowcolor{lulime!60} continue & for    & signed   & void \\
      \rowcolor{lulime}    do       & if     & static   & while \\
      \rowcolor{lulime!60} default  & goto   & sizeof   & volatile \\
      \rowcolor{lulime}    const    & float  & short    & unsigned \\
      \hline
    \end{tabular}
  \end{center}
\end{frame}

\section{Data Types, Variables and Constants}
\begin{frame}{Data Types}
  \begin{description}
    \item[Basic Types:] There are five basic data types
      \begin{enumerate}
        \item int - integer: a whole number.
        \item float - floating point value: ie a number with a fractional part.
        \item double - a double-precision floating point value.
        \item char -  a single character.
        \item void - valueless special purpose type.
      \end{enumerate}
    \item[Derived Types:] These include
      \begin{enumerate}
        \item Pointers
        \item Arrays
        \item Structures
        \item Union
        \item Function
      \end{enumerate}
  \end{description}
  \begin{itemize}
    \item The array and structure types are referred to collectively as the aggregate types. 
    \item The type of a function specifies the type of the function's return value.
  \end{itemize}
\end{frame}

\begin{frame}[fragile]{Basic Data Types: Integer}
  \begin{center}
    \begin{tabular}{|c|c|c|}
      \hline
      Type & Storage size (in bytes) & Value range \\
      \hline
      char            & 1      & -128 to 127 or 0 to 255 \\
      unsigned char   & 1      & 0 to 255 \\
      signed char     & 1      & -128 to 127 \\
      \hline
      \multirow{3}{*}{int} & 2 & -32,768 to 32,767 \\ & or & or \\ & 4 &  -2,147,483,648 to 2,147,483,647 \\
      \hline
      \multirow{3}{*}{unsigned int} & 2 & 0 to 65,535 \\ & or & or  \\ & 4 & 0 to 4,294,967,295 \\
      \hline
      short           & 2      & -32,768 to 32,767 \\
      unsigned short  & 2      & 0 to 65,535 \\ 
      \hline
      long            & 4      & -2,147,483,648 to 2,147,483,647 \\
      unsigned long   & 4      & 0 to 4,294,967,295 \\
      \hline
    \end{tabular}
  \end{center}
  \begin{itemize}
    \item To get the exact size of a type or a variable on a particular platform, you can use the sizeof operator. 
    \item The expressions \lstinline|sizeof(type)| yields the storage size of the object or type in bytes. 
  \end{itemize}
\end{frame}

\begin{frame}{Basic Data Types: Floating-Point \& void}
  \begin{center}
    \begin{tabular}{|c|c|c|c|}
      \hline
      Type & Storage size & Value range & Precision (decimal places) \\
      \hline
      float       & 4 bytes  & 1.2E-38 to 3.4E38     & 6 \\
      double      & 8 bytes  & 2.3E-308 to 1.7E308   & 15 \\
      long double & 10 bytes & 3.4E-4932 to 1.1E4932 & 19 \\
      \hline
    \end{tabular}
  \end{center}

  \begin{center}
    \begin{tabular}{|c|c|}
      \hline
      Situation & Description \\
      \hline
      function returns as void & function with no return value \\
      function arguments as void & function with no parameter \\
      pointers to void & address of an object without type \\
      \hline
    \end{tabular}
  \end{center}
\end{frame}

\begin{frame}[fragile]{Variables}
  \begin{itemize}
    \item Variables are memory location in computer's memory to store data.
    \item To indicate the memory location, each variable should be given a unique name called identifier. 
    \item Variable names are just the symbolic representation of a memory location.
    \item Rules for variable names:
    \begin{enumerate}
      \item Composed of letters (both uppercase and lowercase letters), digits and underscore '\_' only.
      \item The first letter of a variable should be either a letter or an underscore.
      \item There is no rule for the length of a variable name.
      \begin{itemize}
        \item Most likely your code will be used by someone else, so variable names should be meaningful and short as possible.
      \end{itemize}
    \end{enumerate}
    \begin{lstlisting}
int num;
float circle_area;
double _volume;
    \end{lstlisting}
    \item In C programming, you have to declare variable before using it in the program.
  \end{itemize}
\end{frame}

\begin{frame}[fragile]{Declaring Variable or Variable Definition}
  \begin{itemize}
    \item A variable definition means to tell the compiler where and how much to create the storage for the variable. 
    \item A variable definition specifies a data type and contains a list of one or more variables of that type as follows:
    \item[] \lstinline{type variable_list;}
    \item \lstinline{type} must be a valid C data type or any user-defined object, etc., and 
    \item[] \lstinline{variable_list} may consist of one or more identifier names separated by commas.
    \item Variables can be initialized (assigned an initial value) in their declaration.
    \item[] \lstinline{type variable_name = value;}
  \end{itemize}
  \begin{lstlisting}
int    i, j, k;
char   c, ch;
float  f, salary;
double d;
int d = 3, f = 5;           // definition and initializing d and f. 
byte z = 22;                // definition and initializes z. 
char x = 'x';               // the variable x has the value 'x'.
  \end{lstlisting}
\end{frame}

\begin{frame}{Constants \& Literals}
  The constants refer to fixed values that the program may not alter during its execution. These fixed values are also called literals.
  {\footnotesize
    \begin{columns}
      \column{0.45\textwidth}
      \begin{block}{Integer Constants}
        \begin{tabular}{ll}
          85         & /* decimal */\\
          0213       & /* octal */\\
          0x4b       & /* hexadecimal */\\
          30         & /* int */\\
          30u        & /* unsigned int */\\
          30l        & /* long */\\
          30ul       & /* unsigned long */\\
        \end{tabular}
      \end{block}
      \begin{block}{Character Constants}
        \begin{tabular}{ll}
          'a'                               & /* character 'a' */\\
          'Z'                               & /* character 'Z' */\\
          \textbackslash?                   & /*? character */\\
          \textbackslash{}\textbackslash{}  & /*\textbackslash{} character */\\
          \textbackslash{}n                 & /*Newline */\\
          \textbackslash{}r                 & /*Carriage return */\\
          \textbackslash{}t                 & /*Horizontal tab */\\
        \end{tabular}
      \end{block}
      \column{0.45\textwidth}
      \begin{block}{Floating Point Constants}
        \begin{tabular}{ll}
          3.1416    & \\
          314159E-5 & /* 3.14159 */\\
          2.1E+5    & /* 2.1x$10^5$*/\\
          3.7E-2    & /* 0.037 */\\
          0.5E7     & /* 5.0x$10^6$*/\\
          -2.8E-2   & /* -0.028 */\\
        \end{tabular}
      \end{block}
        \begin{block}{String Constants}
          \begin{tabular}{ll}
            "hello, world"  & /* normal string */\\
            "c programming \textbackslash{} & \multirow{2}{*}{/* multi-line string */}\\
            language" & \\
          \end{tabular}
      \end{block}
    \end{columns}
  }
\end{frame}

\begin{frame}[fragile]{How to define Constants}
  \begin{itemize}
    \item Constants can be defined in two ways
    \begin{enumerate}
      \item Using the \lstinline{#define} preprocessor (defining a macro)
      \item Using the \lstinline{const} keyword (new standard borrowed from C++)
    \end{enumerate}
  \end{itemize}
  \lstinputlisting[language=C]{./Example/const.c}
\end{frame}

\begin{frame}{Input and Output}
  \begin{itemize}
    \item C or any programming language in general needs to be interactive i.e. write something back and optionally read data to be useful.
    \item Similar to Unix, C treats all devices as files.
      \begin{center}
        \begin{tabular}{ccc}
          \hline
          Standard File & File Pointer & Device \\
          \hline
          Standard Input & stdin & Keyboard \\
          Standard Output & stdout & Screen \\
          Standard Error & stderr & Screen\\
          \hline
        \end{tabular}
      \end{center}
      
    \item C Programming language provides three functions to read/write from standard input/output
      \begin{center}
        \begin{tabular}{|c|c|c|c|}
          \hline
          & \multicolumn{2}{c|}{Unformatted} & Formatted \\
          \hline
          Input & getchar & gets & scanf \\
          Output & putchar & puts & printf \\
          \hline
        \end{tabular}
      \end{center}
  \end{itemize}
\end{frame}

\begin{frame}[fragile]{Unformatted I/O}
  \begin{block}{The getchar() \& putchar() functions}
    \begin{itemize}
      \item The \lstinline|int getchar(void)| function reads the next available character from the screen and returns it as an integer. 
      \item[] This function reads only single character at a time.
      \item The \lstinline|int putchar(int c)| function puts the passed character on the screen and returns the same character. 
      \item[] This function puts only single character at a time. 
    \end{itemize}
  \end{block}

  \begin{block}{The gets() \& puts() functions}
    \begin{itemize}
      \item The \lstinline|char *gets(char *s)| function reads a line from stdin into the buffer pointed to by s until either a terminating newline or EOF.
      \item The \lstinline|int puts(const char *s)| function writes the string s and a trailing newline to stdout.
    \end{itemize}
  \end{block}
\end{frame}

\begin{frame}[fragile]
  \begin{columns}
    \column{0.5\textwidth}
    \lstinputlisting[language=C]{./Example/readiounfmt1.c}
    \column{0.5\textwidth}
    \lstinputlisting[language=C]{./Example/readiounfmt2.c}
  \end{columns}
\end{frame}

\begin{frame}[fragile]{Formatted I/O}
  \begin{itemize}
    \item The \lstinline|int scanf(const char *format, ...)| function reads input from the standard input stream stdin and scans that input according to format provided.
    \item The \lstinline|int printf(const char *format, ...)| function writes output to the standard output stream stdout and produces output according to a format provided (optional).
      \lstinputlisting[language=C]{./Example/helloworld.c}
    \item In this program, the user is asked a input and value is stored in variable \lstinline|name|.
    \item Note the '\lstinline|&|' sign before \lstinline|name|.
    \item \lstinline|&name| denotes the address of \lstinline|name| and value is stored in that address.
  \end{itemize}
\end{frame}

\begin{frame}[fragile]{Common Format Specifier}
  \begin{itemize}
    \item The format specifier: \lstinline|%[flags][width][.precision][length]specifier| 
  \end{itemize}
  \begin{center}
    \begin{tabular}{cl}
      \hline
      flag & meaning \\
      \hline
      - & left justify \\
      + & always display sign\\
      0 & pad with leading zeros\\
      \hline
    \end{tabular}
  \end{center}
  \begin{center}
    \begin{tabular}{|c|c|c|}
      \hline
      Specifier & Output & Example\\
      \hline
      \%f & decimal float & 3.456 \\
      \hline
      \%7.5f & decimal float, 7 digit width and 5 digit precision & 3.45600 \\
      \hline
      \%d & integer & 5\\
      \hline
      \%05d & integer, 5 digits pad with zeros & 00101 \\
      \hline
      \%s & string of characters & "Hello World!"\\
      \hline
      \%e & scientific notation for decimal float & 2.71828e+5  \\
      \hline
      \%c & character &  \\
      \hline
      \textbackslash{}n & insert new line & \\
      \hline
      \textbackslash{}t & insert tab & \\
      \hline
    \end{tabular}
  \end{center}
\end{frame}

\begin{frame}[fragile]{}
  \lstinputlisting[language=C]{./Example/print.c}

  \begin{lstlisting}[language=C]
alexanders-mbp:Example apacheco$ gcc -o print print.c
alexanders-mbp:Example apacheco$ ./print
Characters: a A 
Decimals: 2014 0065
         floats: 3.14160        3.141600        3.141600e+00 
hello world 

  \end{lstlisting}
\end{frame}

\section{Programming Operators}
\begin{frame}{Arithmetic Operators}
  \begin{center}
    \begin{tabular}{ab}
      \hline
      Operator & Meaning \\
      \hline
      +  & addition or unary plus \\
      -  & subtraction or  unary minus \\
      *  & multiplication \\
      /  & division \\
      \% & remainder after division( modulo division) \\
      \hline
    \end{tabular}
    \newline
    \begin{tabular}{abb}
      \hline
      \multicolumn{3}{c}{Other Arithmetic Operators} \\
      \hline
      Operator & Meaning & Same as \\
      \hline
      =   & a=b   & a=b \\
      +=  & a+=b  & a=a+b \\
      -=  & a-=b  & a=a-b \\
      *=  & a*=b  & a=a*b \\
      /=  & a/=b  & a=a/b \\
      \%= & a\%=b & a=a\%b \\
      \hline
    \end{tabular}
  \end{center}
\end{frame}

\begin{frame}{Relational Operators}
  \begin{itemize}
  \item Relational operators checks relationship between two operands.
  \item If the relation is true, it returns value 1 and if the relation is false, it returns value 0.
  \item Relational operators are used in decision making and loops in C programming.
  \end{itemize}
  \begin{center}
    \begin{tabular}{abb}
      \hline
      Operator & Meaning & Example \\
      \hline
      =  & Equal to                 & 5==3 returns false (0) \\
      >  & Greater than             & 5>3 returns true (1) \\
      <  & Less than                & 5<3 returns false (0) \\
      != & Not equal to             & 5!=3 returns true(1) \\
      >= & Greater than or equal to & 5>=3 returns true (1) \\
      <= & Less than or equal to    & 5<=3 return false (0) \\
      \hline
    \end{tabular}
  \end{center}
\end{frame}

\begin{frame}[fragile]{Logical \& Conditional Operators}
  \begin{itemize}
  \item Logical operators are used to combine expressions containing relation operators.
  \item In C, there are 3 logical operators
  \end{itemize}
  \begin{center}
    \begin{tabular}{abb}
      \hline
      Operator & Meaning & Example \\
      \hline
      \&\& & Logial AND  & If c=5 and d=2 then,((c==5) \&\& (d>5)) returns false. \\
      ||   & Logical OR  & If c=5 and d=2 then, ((c==5) || (d>5)) returns true. \\
      !    & Logical NOT & If c=5 then, !(c==5) returns false. \\
      \hline
    \end{tabular}
  \end{center}
  \begin{itemize}
  \item Conditional Operator: Conditional operators are used in decision making in C programming, i.e, executes different statements according to test condition whether it is either true or false.
  \item[] \lstinline|conditional_expression?expression1:expression2|
  \item If the test condition is true, expression1 is returned and if false expression2 is returned.
  \end{itemize}
\end{frame}


\section{Control Structures: for, if \& switch}
\begin{frame}{Title}

\end{frame}

\end{document}

