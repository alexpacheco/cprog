\documentclass[10pt,t]{beamer}

\include{mypreamble}
\beamertemplateballitem

\newcolumntype{a}{>{\columncolor{lulime}}c}
\newcolumntype{b}{>{\columncolor{lulime!60}}c}

\hypersetup{
  pdftitle={C Programming}
  pdfauthor={Alexander B. Pacheco, LTS Research Computing, Lehigh University}
}

\title{C Programming II}


\author[Alex Pacheco]{\large{Alexander~B.~Pacheco}}
       
\institute{\href{http://researchcomputing.lehigh.edu}{LTS Research Computing}}

\date{June 3, 2015}
     
\subject{Talks}
\keywords{Lehigh Research Computing Resources, C Programming}
% This is only inserted into the PDF information catalog. Can be left
% out. 


% Delete this, if you do not want the table of contents to pop up at
% the beginning of each subsection:
\AtBeginSection[]
{
  \begingroup
  \setbeamertemplate{background canvas}[vertical shading][bottom=lubrown,top=lubrown]
  \setbeamertemplate{footline}[myfootline] 
  \setbeamertemplate{section page}[mysection]
  \frame[c]{
    \sectionpage
  }
  \endgroup
}

\titlegraphic{\includegraphics[scale=0.5]{lu}}

\begin{document}

\frame{\titlepage}

\begin{frame}{Outline}
  \tableofcontents
\end{frame}

\section{Functions}
\begin{frame}[fragile]{Functions}
  \begin{itemize}
  \item A function is a group of statements that together perform a task.
  \item Every C program has at least one function, which is main()
  \item Functions receive either a fixed or variable amount of arguments.
  \item Functions can only return one value, or return no value (void).
  \item In C, arguments are \textbf{passed by value} to functions
  \item How to return value? - \textbf{Pointers}
  \item Functions are defined using the following syntax:
    \begin{lstlisting}
      return_type function_name( parameter list )
      {
        body of the function
      }
    \end{lstlisting}
  \item A function \textbf{declaration} tells the compiler about a function's name, return type, and parameters.
  \item A function \textbf{definition} provides the actual body of the function.
  \end{itemize}
\end{frame}

\begin{frame}[fragile]{Function Definition}
  \begin{itemize}
  \item \textbf{Return Type:} Function's return type is the data type of the value the function returns. When there is no return value, return void.
  \item \textbf{Function Name:} This is the actual name of the function.
  \item \textbf{Parameter:} The parameter list refers to the type, order, and number of the parameters of a function. A function may contain no parameters.
  \item \textbf{Function Body:} The function body contains a collection of statements that define the function behavior.
  \end{itemize}
  \lstinputlisting[language=C,firstline=20,lastline=32]{./Example/findmax.c}
\end{frame}

\begin{frame}{Example of using a Function}
  \lstinputlisting[language=C,basicstyle=\fontsize{5}{6}\selectfont\ttfamily]{./Example/findmax.c}
\end{frame}

\begin{frame}[fragile,allowframebreaks]{Scope Rules: Local \& Global Variables}
  \begin{itemize}
  \item A scope is a region of the program where a defined variable can have its existence and beyond that variable can not be accessed.
  \item \textbf{\color{lublue}Local Variables:} declared inside a function or block.
  \item[] can be used only by statements that are inside that function or block of code.
  \item[] Local variables are not known to functions outside their own.
  \item \textbf{\color{lublue}Global Variables:}  defined outside of a function, usually on top of the program.
  \item[] will hold their value throughout the lifetime of your program and,
  \item[] they can be accessed inside any of the functions defined for the program.
  \item A program can have same name for local and global variables but value of local variable inside a function will take preference.
%  \item \textbf{\color{lublue}Formal Parameters:} Function parameters, formal parameters, are treated as local variables with-in that function and they will take preference over the global variables.
  \end{itemize}
  \lstinputlisting[language=C,basicstyle=\fontsize{5}{6}\selectfont\ttfamily]{./Example/scope.c}
  \begin{lstlisting}[basicstyle=\fontsize{5}{6}\selectfont\ttfamily]
    value of a in main() = 10
    value of a in sum() = 10
    value of b in sum() = 20
    value of c in main() = 30
  \end{lstlisting}
\end{frame}

\begin{frame}{Initializing Local \& Global Variables}
  \begin{itemize}
  \item Local Variables are not initialized by the system, the programmer must initialize it.
  \item Global variables are automatically initialized by the system depending on the data type
    
    \begin{tabular}{ab}
      \rowcolor{lublue}Data Type & Initial Default Value \\
      int & 0 \\
      char & '\textbackslash{}0' \\
      float & 0 \\
      double & 0 \\
      pointer & NULL \\
    \end{tabular}
    \item \textit{It is a good programming practice to initialize variables properly otherwise, your program may produce unexpected results because uninitialized variables will take some garbage value already available at its memory location.}
  \end{itemize}
\end{frame}

\section{Arrays}
\section{Pointers}
\section{Input/Output}

\end{document}

