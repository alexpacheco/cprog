\documentclass[10pt,t]{beamer}

\include{mypreamble}
\beamertemplateballitem

\hypersetup{
  pdftitle={C Programming}
  pdfauthor={Alexander B. Pacheco, LTS Research Computing, Lehigh University}
}

\title{C Programming I}


\author[Alex Pacheco]{\large{Alexander~B.~Pacheco}}
       
\institute{\href{http://researchcomputing.lehigh.edu}{LTS Research Computing}}

\date{\today}
     
\subject{Talks}
\keywords{Lehigh Research Computing Resources, C Programming}
% This is only inserted into the PDF information catalog. Can be left
% out. 


% Delete this, if you do not want the table of contents to pop up at
% the beginning of each subsection:
\AtBeginSection[]
{
  \begingroup
  \setbeamertemplate{background canvas}[vertical shading][bottom=lubrown,top=lubrown]
  \setbeamertemplate{footline}[myfootline] 
  \setbeamertemplate{section page}[mysection]
  \frame[c]{
    \sectionpage
  }
  \endgroup
}

\titlegraphic{\includegraphics[scale=0.5]{lu}}

\begin{document}

\frame{\titlepage}

\begin{frame}{Outline}
  \tableofcontents
\end{frame}

\section{Introduction}
\begin{frame}{What is the C Language?}
  \begin{itemize}
    \item A general-purpose, procedural, imperative computer programming language.
    \item Developed in 1972 by Dennis M. Ritchie at the Bell Telephone Laboratories to develop the UNIX operating system.
    \item The UNIX operating system, the C compiler, and essentially all UNIX applications programs have been written in C.
    \item C is the most widely used computer language.
    \begin{itemize}
      \item Easy to learn
      \item Structured language
      \item Produces efficient programs
      \item Handles low-level activities
      \item Can be compiled on a variety of computer plaforms
    \end{itemize}
    \item Most of the state-of-the-art softwares have been implemented using C.
    \item Today's most popular Linux OS and RBDMS MySQL have been written in C.
  \end{itemize}
\end{frame}

\begin{frame}{What do you need to learn C?}
  \begin{enumerate}
    \item {C Compiler}
      \begin{itemize}
        \item What is a Compiler?
          \begin{itemize}
            \item A compiler is a computer program (or set of programs) that transforms source code written in a programming language (the source language) into another computer language (the target language, often having a binary form known as object code).
          \end{itemize}
        \item How does a compiler do?
          \begin{itemize}
            \item Translate C source code into a binary executable
          \end{itemize}
        \item List of Common Compilers:
          \begin{itemize}
            \item GCC GNU Project (Free, available on most *NIX systems)
            \item Intel Compiler
            \item Portland Group (PGI) Compiler
            \item Microsoft/Borland Compiler
            \item IBM XL Compiler
          \end{itemize}
      \end{itemize}
    \item {Text Editor}
      \begin{itemize}
        \item Emacs
        \item VI/VIM
        \item Notepad++ (avoid Notepad if you will eventually use a *NIX system)
        \item Integrated Development Environment: Eclipse, Visual Studio, etc
      \end{itemize}
  \end{enumerate}
\end{frame}

\begin{frame}[fragile]{My First C Code}
  \begin{itemize}
    \item Simple Hello World Code
      \lstinputlisting[language=C]{./Exercise/hello.c}
    \item Compile using GCC compiler or any other compiler that you prefer
      \begin{lstlisting}[style=LINUX]
dyn100077:Exercise apacheco$ gcc hello.c 
      \end{lstlisting}
    \item Run the compiled binary, a.out
      \begin{lstlisting}[style=LINUX]
dyn100077:Exercise apacheco$ ./a.out 
Hello World!
      \end{lstlisting}
  \end{itemize}
\end{frame}

\section{Basic Syntax}
\section{Program Structure}
\section{Data Types, Variables and Constants}
\section{Control Structures: for, if \& switch}

\end{document}

